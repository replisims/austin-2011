\documentclass[10,a4paperpaper,]{article}

  \title{Replication Report}
  \author{Primary Replicator\textsuperscript{1}, \and Co-pilot
\textsuperscript{2}}
  \date{%
		\textsuperscript{1} University of someplace\\%
		\textsuperscript{2} University of some other place~\\[2ex]
		\today
   }
  


\newcommand{\iblue}{008080}
\newcommand{\igray}{d4dbde}

\newlength{\cslhangindent}
\setlength{\cslhangindent}{1.5em}
\newenvironment{CSLReferences}%
  {}%
  {\par}

\include{defs}


\begin{document}


\renewcommand{\contentsname}{Table of Contents}

\renewcommand{\pagename}{Page}


\urlstyle{same}

\maketitle

\subsection*{Abstract}

\texttt{\textless{}a\ summary\ of\ the\ replication\ effort\textgreater{}}
\vskip 2em

\noindent\makebox[\textwidth]{\large Correspondence concerning this replication report should be addressed to:}

\par

\noindent\makebox[\textwidth]{\large primary\_replicator@someuni.edu}

\par

\clearpage

\section{Introduction}

This replication report documents the replication attempt of the
simulation study Austin, P. C. (2011). Optimal caliper widths for
propensity-score matching when estimating differences in means and
differences in proportions in observational studies. Pharmaceutical
Statistics, 10(2), 150--161. \url{https://doi.org/10.1002/pst.433} .
Following the definition of Rougier et al. (2017) we understand the
replication of a published study as writing and running new code based
on the description provided in the original publication with the aim of
obtaining the same results.

\section{Method}

\subsection{Information basis}

The information for implementing the simulation code stemmed from the
original publication which did not contain any supplemental material.

\subsection{Data Generating Mechanism}

Information provided in the original manuscript indicated that the
following simulation factors were/ varied in generating the artificial
data.

\begin{longtable}[]{@{}
  >{\raggedright\arraybackslash}p{(\columnwidth - 4\tabcolsep) * \real{0.4000}}
  >{\raggedright\arraybackslash}p{(\columnwidth - 4\tabcolsep) * \real{0.1333}}
  >{\raggedright\arraybackslash}p{(\columnwidth - 4\tabcolsep) * \real{0.4667}}@{}}
\toprule()
\begin{minipage}[b]{\linewidth}\raggedright
Simulation factor
\end{minipage} & \begin{minipage}[b]{\linewidth}\raggedright
No.~levels
\end{minipage} & \begin{minipage}[b]{\linewidth}\raggedright
Levels
\end{minipage} \\
\midrule()
\endhead
\emph{Varied} & & \\
Outcome type & 2 & continuous, binary \\
Correlation between covariates & 2 & uncorrelated, pairwise correlation
of 0.25 \\
Covariate type & standard normal, Bernoulli random variable with
parameter 0.5 & \\
Effect size & 5 & for binary outcomes: True risk difference =
0/-0.02/-0.05/-0.10//-0.15 \\
for continuous outcomes: True mean difference = 0/ 1.1/ 1.25/ 1.5/ 2 &
& \\
& & \\
\bottomrule()
\end{longtable}

\emph{Fixed} \textbar\textbar\textbar{}

\subsubsection{Simulation Factor 1}

\texttt{\textless{}More\ detail\ of\ how\ factor\ 1\ was\ varied\ and\ implemented\textgreater{}}

\subsubsection{Simulation Factor 2}

\texttt{\textless{}More\ detail\ of\ how\ factor\ 2\ was\ varied\ and\ implemented\textgreater{}}

\subsubsection{Simulation Factor 3}

\texttt{\textless{}More\ detail\ of\ how\ factor\ 3\ was\ varied\ and\ implemented\textgreater{}}

\texttt{\textless{}If\ the\ design\ was\ not\ full-factorial\ describe\ the\ simulation\ scenarios\ by\ other\ means\textgreater{}}

The simulation design was not full factorial. The authors \ldots{} ten
scenarios, five with a continuous outcome and five corresponding
scenarios with a binary outcome. The author refers to all but the
independent normal covariates scenario as a ``sensitivity analysis''
(p.~152).

\subsubsection{Scenario 1c: Continous outcome Independant normal convariates}

\subsubsection{Scenario 2c: Correlated normal covariates}

\texttt{\textless{}You\ can\ add\ pseudocode\ or\ a\ flowchart\ to\ illustrate\ the\ data\ generation\ or\ the\ entire\ simulation\ design\textgreater{}}

\begin{minipage}{\linewidth}
Data generation can be summarized with the following pseudo code:

\texttt{For 1000 repetitions of each of 400 unique scenarios:}
\begin{itemize}[leftmargin=*] 
    \item[--] \texttt{Set unique seed based on scenario id and number of repetition.}
    \item[--] \texttt{...}
    \item[--] \texttt{ While some condition < some other condition}
    \begin{itemize}
      \item[$\ast$] \texttt{Sample ... from the given distribution.}
      \item[$\ast$] \texttt{Sample a sample size from the given distribution.}
      \item[$\ast$] \texttt{Compute ... based on these random elements.}
      \item[$\ast$] \texttt{Determine ... based on mechanism of current scenario.}
    \end{itemize}
    \item[--] \texttt{If some condition is > x:}
    \begin{itemize}
      \item[$\ast$] \texttt{Determine ... \& resample from corresponding ... model.}
    \end{itemize}
    \item[--] \texttt{Apply ...}
\end{itemize}
\end{minipage}
\newpage

\FloatBarrier 

\subsection{Investigated Methods}

The study investigates the effect of caliper width in the context of
propensity score matching for estimating differences in mean (for
continuous outcomes) and risk differences (for binary outcomes).

Propensity-score matching as implemented in the context of the
investigated study comprised ``pair-matching without replacement within
a specified caliper distance {[}\ldots{]}. Using this approach, pairs of
treated and untreated subjects are formed such that the difference in
propensity scores between matched subjects differs by at most a fixed
distance (the caliper width). In matching without replacement, each
subject can be included in at most one matched set.'' (p.150)

The study ``matched subjects on the logit of the propensity score using
a caliper width of \(\gamma \sqrt{(\sigma_1^2 + \sigma_2^2)/2}\) where
\(\sigma_i^2\) is the variance of the logit of the propensity score in
the \(i\)th group.'' (p.) \(\gamma\) different propensity-score was
allowed to ``range from 0.05 to 2.5 in increments of 0.05. Thus, 50
matched samples were formed from each randomly generated data set.''

\texttt{\textless{}Describe\ the\ methods\ that\ are\ investigated\ and\ how\ they\ are\ implemented\textgreater{}}

\subsubsection{Method 1}

\texttt{\textless{}Describe\ how\ the\ first\ method\ is\ defined\ and\ implemented.\ You\ can\ include\ equations\ and\ or\ R\ code.\ If\ applicable,\ mention\ specialized\ R\ packages,\ their\ version\ as\ well\ as,\ parameters\ of\ specific\ functions.\textgreater{}}

\subsubsection{Method 2}

\texttt{\textless{}Describe\ how\ the\ second\ method\ is\ defined\ and\ implemented.\ You\ can\ include\ equations\ and\ or\ R\ code.\ If\ applicable,\ mention\ specialized\ R\ packages,\ their\ version\ as\ well\ as,\ parameters\ of\ specific\ functions.\textgreater{}}

\subsection{Performance measures}

The study investigated ``the impact of caliper width on reduction in
bias, mean squared error (MSE), coverage of confidence intervals, and
type 1 error rates.'' (p.151)

\texttt{\textless{}Describe\ which\ performance\ measures\ are\ compared,\ if\ applicable\ mention\ specialized\ R\ packages,\ their\ versions,\ as\ well\ as\ parameter\ settings\ of\ functions.\textgreater{}}

The following flowchart describes the simulation design

\begin{figure}
\includegraphics[width=450pt]{flowchart} \caption{Flow chart of data generating mechanism}\label{fig:unnamed-chunk-1}
\end{figure}

\subsection{Technical implementation}

While the original simulation study was carried out in
\texttt{\textless{}computational\ environment\ of\ original\ study\textgreater{}}
our replication was implemented using the R programming environment
(details regarding software versions can be obtained from the section
Reproducibility Information). The corresponding R code can be obtained
from \url{https://github.com/replisims/your_repo}.

The following table provides an overview of replicator degrees of
freedom, i.e.~decisions that had to be made by the replicators because
of insufficient or contradicting information. Issues were resolved by
discussion among the replicators. Decisions were based on what the
replicators perceived to be the most likely implementation with
likeliness estimated by common practice and/or guideline
recommendations. Wherever feasible multiple interpretations where
implemented.

\begin{longtable}[]{@{}
  >{\raggedright\arraybackslash}p{(\columnwidth - 4\tabcolsep) * \real{0.3636}}
  >{\raggedright\arraybackslash}p{(\columnwidth - 4\tabcolsep) * \real{0.3636}}
  >{\raggedright\arraybackslash}p{(\columnwidth - 4\tabcolsep) * \real{0.2727}}@{}}
\toprule()
\begin{minipage}[b]{\linewidth}\raggedright
Issue
\end{minipage} & \begin{minipage}[b]{\linewidth}\raggedright
Replicator decision
\end{minipage} & \begin{minipage}[b]{\linewidth}\raggedright
Justification
\end{minipage} \\
\midrule()
\endhead
Data dependence & each scenario is implemented in independently
generated data & Best practice (Burton et al. 2006) \\
\bottomrule()
\end{longtable}

\subsection{Some issue}

\texttt{\textless{}More\ details\ on\ how\ the\ information\ provided\ was\ insufficient,\ unclear\ or\ vague\textgreater{}}
\emph{``Some weird quote from the original article that you could not
make any sense of''} (p.XY)

\subsection{Another issue}

\texttt{\textless{}More\ details\ on\ how\ the\ information\ provided\ was\ insufficient,\ unclear\ or\ vague\textgreater{}}
\emph{``Some weird quote from the original article that you could not
make any sense of''} (p.XY)

\section{Results}

\subsection{Simulation descriptives}

\texttt{\textless{}Describe\ the\ sampling\ distribution\ if\ any\ of\ the\ simulation\ parameters\ were\ sampled\textgreater{}}

\subsection{Replication of result figures}
\subsubsection{Outcome 1}

\texttt{\textless{}Provide\ the\ information\ that\ were\ presented\ in\ figure\ form\ in\ the\ original\textgreater{}}

\subsection{Replication of result tables}

\texttt{\textless{}Compare\ any\ tabulated\ data\ to\ the\ original\textgreater{}}

\subsection{Replication of results presented in text form }

\texttt{\textless{}If\ the\ text\ describes\ any\ results\ using\ words\ describe\ how\ that\ relates\ to\ your\ findings.\textgreater{}}

\FloatBarrier
\section{Discussion}

\subsection{Replicability}

\texttt{\textless{}Provide\ a\ general\ statement\ of\ how\ you\ experienced\ the\ replication\ process.\ Was\ it\ easy?\ What\ made\ it\ easy\ or\ difficult?\textgreater{}}

\subsection{Replicator degrees of freedom}

\texttt{\textless{}Here\ you\ can\ discuss\ the\ replicator\ degrees\ of\ freedom.\ What\ could\ the\ authors\ have\ done\ to\ make\ it\ more\ clear?\ Do\ you\ think\ the\ replicator\ degrees\ of\ freedom\ are\ so\ extensive\ that\ they\ could\ influence\ the\ results?\textgreater{}}

\subsection{Equivalence of results}

\texttt{\textless{}How\ would\ you\ judge\ the\ overall\ equivalence\ of\ results?\ Are\ the\ orders\ of\ magnitude\ comparable?\ Are\ trends\ in\ the\ same\ direction?\ Would\ you\ draw\ the\ same\ conclusions\ as\ the\ authors\ based\ on\ your\ replication?\ Were\ some\ results\ not\ comparable\ because\ of\ insufficient\ figure\ resolution\ or\ labeling?\ Did\ the\ authors\ ommit\ some\ results\ which\ consequently\ cannot\ be\ compared?\textgreater{}}

\section{Acknowledgments}

\texttt{\textless{}Acknowledge\ the\ help\ of\ anyone\ who\ assisted\ you\ in\ the\ process\textgreater{}}

\section{Contributions}

Authors made the following contributions according to the CRediT
framework \url{https://casrai.org/credit/}

Primary Replicator:

\begin{itemize}
\tightlist
\item
  Data Curation\\
\item
  Formal Analysis (lead)\\
\item
  Investigation\\
\item
  Software\\
\item
  Visualization (lead)\\
\item
  Writing - Original Draft Preparation\\
\item
  Writing - Review \& Editing
\end{itemize}

Co-Pilot:

\begin{itemize}
\tightlist
\item
  Formal Analysis (supporting)\\
\item
  Investigation\\
\item
  Software (supporting)\\
\item
  Visualization (supporting)\\
\item
  Validation\\
\item
  Writing - Review \& Editing
\end{itemize}

\newpage

\section*{References}
\begingroup
\hphantom{x}
\setlength{\parindent}{-0.5in}
\setlength{\leftskip}{0.5in}

\hypertarget{refs}{}
\begin{CSLReferences}{1}{0}
\leavevmode\vadjust pre{\hypertarget{ref-burton_design_2006}{}}%
Burton, Andrea, Douglas G. Altman, Patrick Royston, and Roger L. Holder.
2006. {``The Design of Simulation Studies in Medical Statistics.''}
\emph{Statistics in Medicine} 25 (24): 4279--92.
\url{https://doi.org/10.1002/sim.2673}.

\leavevmode\vadjust pre{\hypertarget{ref-rougier_sustainable_2017-1}{}}%
Rougier, Nicolas P., Konrad Hinsen, Frédéric Alexandre, Thomas Arildsen,
Lorena A. Barba, Fabien C. Y. Benureau, C. Titus Brown, et al. 2017.
{``Sustainable Computational Science: The {ReScience} Initiative.''}
\emph{PeerJ Computer Science} 3 (December): e142.
\url{https://doi.org/10.7717/peerj-cs.142}.

\end{CSLReferences}

\FloatBarrier
\endgroup
\newpage

\section*{Appendix}

\subsection*{Additional result}

\texttt{\textless{}insert\ additional\ results\ not\ reported\ in\ the\ original\ article\ or\ results\ presented\ in\ an\ alternative\ way\textgreater{}}

\subsection{Code organization}

The code and the files associated are organized in the form of a
research compendium which can be found in the following git repository
\texttt{https://github.com/replisims/your\_repo}

\begin{verbatim}
## .
## +-- defs.tex
## +-- flowchart.PNG
## +-- Lato-Black.ttf
## +-- Lato-BlackItalic.ttf
## +-- Lato-Bold.ttf
## +-- Lato-BoldItalic.ttf
## +-- Lato-Italic.ttf
## +-- Lato-Regular.ttf
## +-- references.bib
## +-- Replication Report Austin 2010.Rmd
## +-- Replication-Report-Austin-2010.log
## +-- Replication-Report-Austin-2010.pdf
## +-- Replication-Report-Austin-2010.Rmd
## +-- Replication-Report-Austin-2010.tex
## +-- UbuntuMono-Bold.ttf
## +-- UbuntuMono-BoldItalic.ttf
## +-- UbuntuMono-Italic.ttf
## \-- UbuntuMono-Regular.ttf
\end{verbatim}

\begin{itemize}
\tightlist
\item
  \texttt{foldername}: contains
  \texttt{\textless{}insert\ description\textgreater{}}
\item
  \texttt{filename}: contains
  \texttt{\textless{}insert\ description\textgreater{}}
\item
  \ldots{}
\end{itemize}

\subsubsection*{Reproducibility Information}

This report was last updated on 2022-12-05 10:54:51. The simulation
replication was conducted using the following computational environment
and dependencies:

\FloatBarrier

\begin{verbatim}
## - Session info ---------------------------------------------------------------
##  setting  value
##  version  R version 4.1.3 (2022-03-10)
##  os       Windows 10 x64 (build 19045)
##  system   x86_64, mingw32
##  ui       RTerm
##  language (EN)
##  collate  English_United States.1252
##  ctype    English_United States.1252
##  tz       Europe/Berlin
##  date     2022-12-05
##  pandoc   2.18 @ C:/Program Files/RStudio/bin/quarto/bin/tools/ (via rmarkdown)
## 
## - Packages -------------------------------------------------------------------
##  package        * version    date (UTC) lib source
##  assertthat       0.2.1      2019-03-21 [1] CRAN (R 4.1.2)
##  cachem           1.0.6      2021-08-19 [1] CRAN (R 4.1.2)
##  callr            3.7.1      2022-07-13 [1] CRAN (R 4.1.3)
##  cli              3.4.1      2022-09-23 [1] CRAN (R 4.1.3)
##  crayon           1.5.1      2022-03-26 [1] CRAN (R 4.1.3)
##  DBI              1.1.3      2022-06-18 [1] CRAN (R 4.1.3)
##  devtools         2.4.3      2021-11-30 [1] CRAN (R 4.1.2)
##  digest           0.6.29     2021-12-01 [1] CRAN (R 4.1.2)
##  dplyr          * 1.0.9      2022-04-28 [1] CRAN (R 4.1.3)
##  ellipsis         0.3.2      2021-04-29 [1] CRAN (R 4.1.2)
##  evaluate         0.15       2022-02-18 [1] CRAN (R 4.1.3)
##  fansi            1.0.3      2022-03-24 [1] CRAN (R 4.1.3)
##  fastmap          1.1.0      2021-01-25 [1] CRAN (R 4.1.2)
##  fs               1.5.2      2021-12-08 [1] CRAN (R 4.1.2)
##  generics         0.1.3      2022-07-05 [1] CRAN (R 4.1.3)
##  glue             1.6.2      2022-02-24 [1] CRAN (R 4.1.2)
##  htmltools        0.5.2      2021-08-25 [1] CRAN (R 4.1.2)
##  knitr          * 1.39       2022-04-26 [1] CRAN (R 4.1.3)
##  lifecycle        1.0.1      2021-09-24 [1] CRAN (R 4.1.2)
##  magrittr         2.0.2      2022-01-26 [1] CRAN (R 4.1.2)
##  memoise          2.0.1      2021-11-26 [1] CRAN (R 4.1.2)
##  pillar           1.8.0      2022-07-18 [1] CRAN (R 4.1.3)
##  pkgbuild         1.3.1      2021-12-20 [1] CRAN (R 4.1.2)
##  pkgconfig        2.0.3      2019-09-22 [1] CRAN (R 4.1.2)
##  pkgload          1.3.0      2022-06-27 [1] CRAN (R 4.1.3)
##  prettyunits      1.1.1      2020-01-24 [1] CRAN (R 4.1.2)
##  processx         3.7.0      2022-07-07 [1] CRAN (R 4.1.3)
##  ps               1.6.0      2021-02-28 [1] CRAN (R 4.1.2)
##  purrr            0.3.4      2020-04-17 [1] CRAN (R 4.1.2)
##  R6               2.5.1      2021-08-19 [1] CRAN (R 4.1.2)
##  remotes          2.4.2      2021-11-30 [1] CRAN (R 4.1.2)
##  RepliSimReport   0.0.0.9000 2022-02-03 [1] Github (replisims/RepliSimReport@5f14003)
##  rlang            1.0.4      2022-07-12 [1] CRAN (R 4.1.3)
##  rmarkdown        2.13       2022-03-10 [1] CRAN (R 4.1.3)
##  rstudioapi       0.13       2020-11-12 [1] CRAN (R 4.1.2)
##  sessioninfo      1.2.2      2021-12-06 [1] CRAN (R 4.1.2)
##  stringi          1.7.8      2022-07-11 [1] CRAN (R 4.1.3)
##  stringr          1.4.0      2019-02-10 [1] CRAN (R 4.1.2)
##  tibble           3.1.8      2022-07-22 [1] CRAN (R 4.1.3)
##  tidyselect       1.1.2      2022-02-21 [1] CRAN (R 4.1.3)
##  usethis          2.1.5      2021-12-09 [1] CRAN (R 4.1.2)
##  utf8             1.2.2      2021-07-24 [1] CRAN (R 4.1.2)
##  vctrs            0.4.1      2022-04-13 [1] CRAN (R 4.1.3)
##  xfun             0.31       2022-05-10 [1] CRAN (R 4.1.3)
##  xtable         * 1.8-4      2019-04-21 [1] CRAN (R 4.1.2)
##  yaml             2.3.5      2022-02-21 [1] CRAN (R 4.1.2)
## 
##  [1] C:/Users/alohmann/Documents/R/win-library/4.1
##  [2] C:/Program Files/R/R-4.1.3/library
## 
## ------------------------------------------------------------------------------
\end{verbatim}

The current Git commit details are:

\begin{verbatim}
## Local:    comments C:/Users/alohmann/Desktop/sort/austin2011/austin2011
## Remote:   comments @ origin (https://github.com/replisims/austin-2011.git)
## Head:     [9644e69] 2022-05-21: Merge branch 'master' into comments
\end{verbatim}


\end{document}
